\documentclass[12pt]{scrartcl}


\setlength{\parindent}{0pt}
\setlength{\parskip}{.25cm}

\usepackage{graphicx}

\usepackage{xcolor}

\definecolor{darkred}{rgb}{0.5,0,0}
\definecolor{darkgreen}{rgb}{0,0.5,0}
\usepackage[pdfusetitle]{hyperref}
\hypersetup{
  letterpaper,
  colorlinks,
  linkcolor=red,
  citecolor=darkgreen,
  menucolor=darkred,
  urlcolor=blue,
  pdfpagemode=none,
  pdftitle={Lab 6.0 - Functions II},
  pdfauthor={Christopher M. Bourke}
}

\definecolor{MyDarkBlue}{rgb}{0,0.08,0.45}
\definecolor{MyDarkRed}{rgb}{0.45,0.08,0}
\definecolor{MyDarkGreen}{rgb}{0.08,0.45,0.08}

\definecolor{mintedBackground}{rgb}{0.95,0.95,0.95}
\definecolor{mintedInlineBackground}{rgb}{.90,.90,1}

%\usepackage{newfloat}
\usepackage[newfloat=true]{minted}
\setminted{mathescape,
               linenos,
               autogobble,
               frame=none,
               framesep=2mm,
               framerule=0.4pt,
               %label=foo,
               xleftmargin=2em,
               xrightmargin=0em,
               startinline=true,  %PHP only, allow it to omit the PHP Tags *** with this option, variables using dollar sign in comments are treated as latex math
               numbersep=10pt, %gap between line numbers and start of line
               style=default, %syntax highlighting style, default is "default"
               			    %gallery: http://help.farbox.com/pygments.html
			    	    %list available: pygmentize -L styles
               bgcolor=mintedBackground} %prevents breaking across pages
               
\setmintedinline{bgcolor={mintedBackground}}
\setminted[text]{bgcolor={mintedBackground},linenos=false,autogobble,xleftmargin=1em}
%\setminted[php]{bgcolor=mintedBackgroundPHP} %startinline=True}
\SetupFloatingEnvironment{listing}{name=Code Sample}
\SetupFloatingEnvironment{listing}{listname=List of Code Samples}

\title{CSCE 155 - C}
\subtitle{Lab 6.0 - Functions, Passing By Reference, and Error Handling}
\author{~}
\date{~}

\begin{document}

\maketitle

\section*{Prior to Lab}

Before attending this lab:
\begin{enumerate}
  \item Read and familiarize yourself with this handout.
  \item Read Chapters 5--6 and 18--19 of the \href{http://cse.unl.edu/~cbourke/ComputerScienceOne.pdf}{Computer Science I} textbook
  \item Watch Videos 6.1 thru 6.5 of the \href{https://www.youtube.com/playlist?list=PL4IH6CVPpTZVkiEnCEOdGbYsFEdtKc5Bx}{Computer Science I} video series
\end{enumerate}

\section*{Peer Programming Pair-Up}

\textbf{For students in the online section:} you may complete
the lab on your own if you wish or you may team up with a partner
of your choosing, or, you may consult with a lab instructor to get
teamed up online (via Zoom).

\textbf{For students in the face-to-face section:} your
lab instructor will team you up with a partner.  

To encourage collaboration and a team environment, labs are be
structured in a \emph{peer programming} setup.  At the start of
each lab, you will be randomly paired up with another student 
(conflicts such as absences will be dealt with by the lab instructor).
One of you will be designated the \emph{driver} and the other
the \emph{navigator}.  

The navigator will be responsible for reading the instructions and
telling the driver what to do next.  The driver will be in charge of the
keyboard and workstation.  Both driver and navigator are responsible
for suggesting fixes and solutions together.  Neither the navigator
nor the driver is ``in charge.''  Beyond your immediate pairing, you
are encouraged to help and interact and with other pairs in the lab.

Each week you should alternate: if you were a driver last week, 
be a navigator next, etc.  Resolve any issues (you were both drivers
last week) within your pair.  Ask the lab instructor to resolve issues
only when you cannot come to a consensus.  

Because of the peer programming setup of labs, it is absolutely 
essential that you complete any pre-lab activities and familiarize
yourself with the handouts prior to coming to lab.  Failure to do
so will negatively impact your ability to collaborate and work with 
others which may mean that you will not be able to complete the
lab.  

\section{Lab Objectives \& Topics}
At the end of this lab you should be familiar with the following
\begin{itemize}
  \item Basics of enumerated types
  \item Functions with pass-by-value and pass-by-reference parameters
  \item How to design and use functions with error handling
  \item Have exposure to a formal unit testing framework 
\end{itemize}

\section{Background}

\subsection*{Enumerated Types}

Enumerated types are data types that define a set of named values.  
Enumerated types are often ordered and internally associated with 
integers (starting with 0 by default and incremented by one in the 
order of the list).  The purpose of enumerated types is to organize 
certain types and enforce specific values.  Without enumerated types, 
integer values must be used and the convention of assigning certain 
integer values to certain types in a logical collection must be remembered 
or documentation referred to as needed.  Enumerated types provide 
a human-readable ``tag'' to these types of elements, relieving the 
programmer of continually having to refer to the convention and avoiding 
errors.

\subsection*{Passing by Value versus Passing by Reference}

In general there are two ways in which arguments can be passed to 
a function: by value or by reference. When an argument (a variable) 
is passed by value to a function, a copy of the value stored in the 
variable is placed on the program stack and the function uses this 
copy's value during its execution.  Any changes to the copy are not 
realized in the calling function, but are local to the function.

In contrast, when an argument (variable) is passed by reference to a 
function, the variable's memory address is used.  Locally (that is, 
\emph{inside the function}), the variable 
is treated as a pointer; to get or set its value, it must be dereferenced 
using the \mintinline{c}{*} operator.  Any changes to the memory cell pointed to by 
that pointer are reflected globally in the program and more specifically, 
in the calling function.

A full example:

\begin{minted}{c}
//a regular variable:
int a = 10; 

//a pointer to an integer, initially pointing to NULL
int *ptr = NULL; 

// & is the referencing operator, it gets the memory 
// address of the variable a so that ptr can point to it
ptr = &a; 

// The star: * is the dereferencing operator.  It changes 
// the pointer back into a regular variable so a value can 
// accessed or assigned:
*ptr = 20; //a now holds the value 20
\end{minted}

One of the main uses of pointers is to enable pass-by-reference 
for functions.  Consider the following function:

\begin{minted}{c}
int foo(int a, int b) {
  a = 10;
  return a + b;
}
\end{minted}

When we call \mintinline{c}{foo}, the variable's we pass to it are passed
\emph{by value} meaning that the value stored in the variables at the
time that we call the function are copied and given to the function.  The
change to the variable \mintinline{c}{a} in line 2 \emph{only affects
the local variable}.  The original variable is unaltered.  That is, the following
code:

\begin{minted}{c}
int x = 5;
int y = 20;
int z = foo(x, y);
printf("x, y, z = %d, %d, %d\n" x, y, z);
\end{minted}

would print \mintinline{text}{5, 20, 30}: the variable \mintinline{c}{x} is not
changed by the function \mintinline{c}{foo}.  Contrast this with the following:

\begin{minted}{c}
int bar(int *a, int b) {
  *a = 10;
  return (*a) + b;
}
\end{minted}

In the function \mintinline{c}{bar}, the variable \mintinline{c}{a} is passed
\emph{by reference} using a pointer.  Since the memory address is passed
to \mintinline{c}{bar} instead of a copy of a value, we can alter the original
value.  Now when line 2 changes the value of the pointer \mintinline{c}{a}, 
the change affects the original variable.  That is, the code 

\begin{minted}{c}
int x = 5;
int y = 20;
int z = foo(&x, y);
printf("x, y, z = %d, %d, %d\n" x, y, z);
\end{minted}

Will now print \mintinline{text}{10, 20, 30} since the variable \mintinline{c}{x}
was passed by reference.

Passing by reference also allows us to design functions that can
compute multiple values and ``return'' them in pass-by-reference
pointer variables.  This frees up the return value to be used as 
an error code that can be communicated back to the calling function
in the event of an error or bad input values.

\section{Activities}

Clone the GitHub project for this lab using the following URL:
\url{https://github.com/cbourke/CSCE155-C-Lab06}

\subsection{(Re)designing Your Functions}

In the previous lab you designed several functions to convert RGB
values to gray-scale (using one of three techniques) and to sepia.
The details of how to do this available in the previous lab and 
are repeated for your convenience below.  

The design of those functions was less than ideal.  For the
gray scale functions, there was a function for each of the three 
techniques.  For the sepia conversion, we had to have three separate
functions (one for each component, red, green, blue).  This was 
necessary because functions can only return one value.  We can't have
a function compute and ``return'' all three RGB values.  However,
if we use pass-by-reference variables, we \emph{can} compute and
``return'' multiple values with a single function!  Moreover, we
can then use the actual return value to indicate an error with the inputs
(if any).  

Redesign your functions from the previous lab as follows.  For 
the gray scale functionality, implement a single function with
the following signature:

\mintinline{c}{int toGrayScale(int *r, int *g, int *b, Mode m)}

The function takes three integer values by reference and will 
use the value stored in them to compute a gray scale RGB value.
It will then store the result \emph{back} into the variables.

To specify which of the three \emph{modes} is to be used, we have
defined an enumerated type in the \mintinline{text}{colorUtils.h}
header file:
\begin{minted}{c}
typedef enum {
  AVERAGE,
  LIGHTNESS,
  LUMINOSITY
} Mode;
\end{minted}
Finally, identify any and all error conditions and use the return 
value to indicate an error code (0 for no error, non-zero value(s) 
for error conditions).  You should define another enumerated type
to represent error codes.

For the sepia filter, implement a single function with the following
signature:

\mintinline{c}{int toSepia(int *r, int *g, int *b);}

The function should use the values stored in the variables passed
by reference and then store the results in them.  Again, error codes
should be returned for invalid input values and you should use the
enumerated type you defined for error codes.

Finally, add proper documentation to your functions' prototypes.

\subsection{Running Unit Tests}

\textbf{NOTE}: if you are using the CS50 IDE you may need to (re)install
cmocka using the following command (and answering \mintinline{text}{Y}
when prompted):

\mintinline{text}{sudo apt-get install libcmocka-dev libcmocka0}

As before, you can test your functions using the full image driver
program.  To build the project use \mintinline{text}{make} and
run the executable \mintinline{text}{imageDriver}, testing it on
a few images of your choice.  This is essentially an 
\emph{ad-hoc test} which is not very rigorous nor reliable 
and is a manual process.  

In the last lab you wrote several \emph{informal} unit tests.  
Writing unit tests automates the testing process and is far more
rigorous.  However, this involved writing a lot of boilerplate 
code to run the tests, print out the results and keep track 
of the number passed/failed.  

In practice, it is better to use a more formal unit testing framework
or library.  There are several such libraries for C, but one that 
we'll use is cmocka (\url{https://cmocka.org/}).  We have provided
unit testing code in a file, \mintinline{text}{colorUtilsTesterCmocka.c} 
that implements and runs a \emph{suite} of unit tests. You can 
build this testing suite using :

\mintinline{text}{make colorUtilsTesterCmocka}

and run the resulting executable:

\mintinline{text}{./colorUtilsTesterCmocka}

This starter file should be sufficient to demonstrate how to 
use cmocka, but the full documentation 
can be found here: \url{https://api.cmocka.org/}.  Note that cmocka 
is already installed on the CSE server.  If you compile on your own 
machine, you will have to install and troubleshoot cmocka yourself.  

Run the test suite and verify that your code passes all the tests.
Fix any issues or bugs that become apparent as a result of this
testing.  Show your 100\% completed program to a lab instructor
and get your worksheet signed off.

Time permitting, add a few of your own test cases to the test
suite.

\section*{Color Formulas}

To convert an RGB value to gray-scale you can use one of several
different techniques.  Each technique ``removes'' the color value by
setting all three RGB values to the same value but each technique 
does so in a slightly different way.

The first technique is to simply take the average of all three values:
  $$\frac{r + g + b}{3}$$

The second technique, known as the ``lightness'' technique averages 
the most prominent and least prominent colors:
  $$\frac{\max\{r, g, b\} + \min\{r, g, b\}}{2}$$

The luminosity technique uses a weighted average to account for a human 
perceptual preference toward green, setting all three values to:
  $$0.21 r + 0.72 g + 0.07 b$$
In all three cases, the integer values should be \emph{rounded} rather 
than truncated.

A sepia filter sets different values to each of the three RGB components 
using the following formulas.  Given a current $(r,g,b)$ value, the sepia
tone RGB value, $(r',g',b')$ would be:
$$\begin{array}{ll}
  r' &= 0.393r + 0.769g + 0.189b \\
  g' &= 0.349r + 0.686g + 0.168b \\
  b' &= 0.272r + 0.534g + 0.131b
\end{array}$$
As with the gray-scale techniques, values should be rounded.  If any of
the resulting RGB values exceeds 255, they should be reset to the 
maximum, 255.

\section{Handin/Grader Instructions}

\begin{enumerate}
  \item Hand in your completed files:
  \begin{itemize}
    \item \mintinline{text}{colorUtils.c}
    \item \mintinline{text}{colorUtils.h}
    \item \mintinline{text}{colorUtilsTesterCmocka.c}
    \item (no worksheet is necessary for this lab)
  \end{itemize}
  through the webhandin (\url{https://cse-apps.unl.edu/handin}) 
  using your cse login and password.  
  \item Even if you worked with a partner, you \emph{both} should
  turn in all files.
  \item Verify your program by grading yourself through the
  webgrader (\url{https://cse.unl.edu/~cse155e/grade/}) using the
  same credentials.
  \item Recall that both expected output and your program's output
  will be displayed.  The formatting may differ slightly which is fine.
  As long as your program successfully compiles, runs and outputs 
  the \emph{same values}, it is considered correct.
\end{enumerate}

\section{Advanced Activity (Optional)}

Cmocka is only one of many possible unit testing frameworks available
for C.  Another example is Google Test 
(\url{https://github.com/google/googletest}).  Read the following tutorial 
\url{http://notes.eatonphil.com/unit-testing-c-code-with-gtest.html} and
rewrite the cmocka unit tests using Google Test's framework/library instead
in order to get exposure to how different unit testing libraries are designed.

\end{document}
